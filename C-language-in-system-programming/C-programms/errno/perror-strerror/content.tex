В реальном коде, для быстрого получения описания ошибки используются функции prerror и strerror. Они работают, анализируя текущее значение errno.

\begin{CCode}{perror (3)}
	void perror(
		const char *str
	); \end{CCode}
Выведет в STDOUT сообщение вида “str:  <описание текущего значения errno>”

\begin{CCode}{strerror (3)}
	char * strerror(
		int errnum
	); \end{CCode}
Вернет указатель на строку с описанием текущего значения errno


Например в нашей исходной функции, perror (3) используется, чтобы диагностировать ошибку, возникающую в функции main, если ОС не удалось открыть файл.

\begin{CCode}{main.c}
	if (open(argv[1], O_RDONLY) < 0) {
		perror("main");			
		return EXIT_FAILURE;
	} \end{CCode}	 