Для того, чтобы помочь пользователю понять, чего ожидает от него интерпретатор, существуют так называемые prompt string (или приглашения командной строки). Все они имеют название по первым буквам - PS1, PS2, PS3, PS4 и являются частью интерактивного режима ввода.

Их можно переопредилить в файлах инициализации (например, .profile).

\begin{myenv}{\$PS1}{выводится в качестве для ввода команд. Туда можно помещать полезную
информацию, такую, как текущий каталог, имя пользователя, под которым вы зашли, и так
далее.}

Очень часто в качестве последних символов в нем можно увидеть “\$ {<пробел>}“ (для обычного пользователя), либо “\# {<пробел>}“ (для пользователя, обладающего root-правами).
\end{myenv}

\begin{myenv}{\$PS2}{выводится в случае, если мы начали вводить команду, но по какой-то причине не
закончили ввод и нажали Enter.}

Самое распространенное значение этой переменной - “{>}“.
\end{myenv}

\begin{myenv}{\$PS3}{выводится тогда, когда оператор select ожидает ввода значений.}
\end{myenv}

\begin{myenv}{\$PS4}{выводится в начале каждой строки вывода, когда сценарий вызывается с ключом -x.}
\end{myenv}