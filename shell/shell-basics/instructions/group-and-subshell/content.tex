\textbf{Фигурные скобки}

Фигурные скобки используются для группировки команд. Перечисленные в \textbf{\{ command1; command2; ... \}} исполняются текущим командным интерпретатором.

Другая возмоможность использования --- список возможных вариантов в \textbf{\{ var1,var2, ... \}}. 

\begin{shCode}{Например}
	ag@helios:/home/ag$ touch file_{a,b,c} \end{shCode}

Создаст в текущем рабочем каталоге файлы с именами file\_a, file\_b, file\_c.

\textbf{Запуск команд в subshell}

При перечислении списка команд в круглых скобках \textbf{\$( <command> )} или обратных кавычках (гравис) \textbf{` <command> `}, они исполняются в дочернем shell (или по-другому subshell или подоболочка). То есть этот список команд будет исполнен новым экземпляром командного интерпретатора.

Отметим, что значения переменных, определенных в subshell, не передаются родительской оболочке и недоступны ей.

\begin{shCode}{Например}
	ag@helios:/home/ag$ ls -l $(echo mydir) \end{shCode}

\begin{shCode}{Или эквивалентное ему}
	ag@helios:/home/ag$ ls -l `echo mydir` \end{shCode}

Выведут содержимое директории mydir --- сначала выполнится команда echo в subshell, потом на место вызова subshell подставится результат работы --- строка “mydir“.