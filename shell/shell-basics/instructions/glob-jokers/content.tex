В shell есть три glob-джокера. Это \textbf{“*”}, \textbf{“?”}, \textbf{“[]”}. Будьте осторожны и не путайте их с квантификаторами регулярных выражений. 

Glob-джокеры --- это специальные символы-шаблоны поиска, которые в некоторых случаях заменяются shell, в зависимости от их назначения. 

\begin{myenv}{*}{означает любое количество любых символов}
\end{myenv}
\begin{myenv}{?}{означает один любой символ}
\end{myenv}
\begin{myenv}{[]}{работает как символьный класс (вхождение одного из символов, перечисленного в скобках). Если наоборот нужно исключить какие-либо символы, после [ можно поставить символ \textasciicircum}
\end{myenv}

\begin{shCode}{Например}
	ag@helios:/home/ag$ ls mydir/f?l*.sh \end{shCode}

Выведет все файлы, содержащиеся в директории mydir, которые содержат последовательность “f“, один произвольный символ, символ “l“, любое количество любых символов, последовательность символов “.sh“.