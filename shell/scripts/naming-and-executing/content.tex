В случае shell, расширение --- атавизм из Windows. То есть для shell .sh, .bash, .ksh --- не более, чем окончание имени файла. 
Тем не менее, добавлять к имени файла .sh (или название другого shell), можно для того, чтобы пользователь понимал, что это shell-скрипт. Например, если планируется передавать этот скрипт широкому кругу пользователей.

Сохранив скрипт, можно запустить его следующим образом:

\begin{shCode}{ }
		ag@helios:/home/ag$ chmod 755 myscript
		ag@helios:/home/ag$ ./myscript 1 2 3 \end{shCode}
		
\begin{important}
	Перед первым запуском не забудьте дать себе права на выполнение этого скрипта с помощью chmod.
\end{important}