Программы, написанные на языке командного интерпретатора, называют сценариями или скриптами. 

Скрипт может быть передан командному интерпретатору с помощью специального командного файла. Особенностью этого специального файла будет являться наличие в первой строке конструкции вида:
\textbf{\#!/bin/sh} (\#!<путь к командному интерпретатору, из которого мы хотим запустить скрипт).

Первые два символа называются shebang. Есть две возможных трактовки этого названия. Первая --- от слова Shell, второй вариант ближе к символьной трактовке: Sharp(\#) и bang(!).