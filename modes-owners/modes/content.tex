У файла есть три группы модификаторов:

\begin{itemize}
	\item права владельца --- обозначаются u (user);

	\item права для членов группы владельца --- обозначаются g (group);

	\item права для всех остальных --- обозначаются o (other).
\end{itemize}

Пользовательские права доминируют над групповыми. Если кто-то другой принадлежит группе, владеющей файлом, то права группы рассматриваются с большим приоритетом, чем права для остальных. 

\textbf{Модификаторы доступа к файлам.}

Существуют три основных модификатора доступа: чтение (r — read), запись (w — write) и выполнение (x - eXecute). Установленный модификатор дает пользователю, группе или остальным соответствующее этому модификатору право.

Модификаторы доступа для файлов и жестких ссылок дают возможность:

\begin{itemize}
	\item \textbf{r} — просматривать содержимое;
	\item \textbf{w} — записывать/редактировать файл;
	\item \textbf{x} — попытаться выполнить как программу.
\end{itemize}

Модификаторы доступа для каталога дают возможность:

\begin{itemize}
	\item \textbf{r} — просматривать список содержащихся файлов;
	\item \textbf{w} — создавать файлы и каталоги внутри этого каталога;
	\item \textbf{x} — войти в каталог.
\end{itemize}
Для символических ссылок права определяются правами файла, на которой они указывают.


Также существуют три специальных бита SUID, SGID и Sticky bit: 

\begin{itemize}
	\item \textbf{SUID (Set User ID)} — позволяет пользователю запустить исполняемый файл от имени настоящего владельца файла (с правами владельца);

	\item \textbf{SGID (Set Group ID)} — позволяет пользователю запустить исполняемый файл от имени группы настоящего владельца файла (с правами группы);

	\item \textbf{Sticky bit} — разрешает всем пользователям писать в файл или каталог, но право удаления остается только за владельцем.
\end{itemize}


\textbf{Списки контроля доступа.}

\begin{defi}{ACL (Access Control Lists)}
	расширенные списки контроля доступа.
\end{defi}

Это дополнительная информация о файлах, которая позволяет конкретным отдельным пользователям или группам задать другие своеобразные специфические права. 