У файлов можно менять доступ, и для этого у нас есть два системных вызова: chmod (2) и fchmod (2). Отличаются они первым аргументом. В первом случае, это имя файла, у которого мы хотим поменять права, во втором случае, это файловый дескриптор открытого файла. Второй аргумент одинаковый и представляет из себя четырехразрядное восьмеричное число --- желаемый режим доступа.

\begin{CCode}{chmod (2)}
	int chmod(
		const char *path, 
		mode_t mode
	); \end{CCode}

\begin{CCode}{fchmod (2)}
	int fchmod(
		int fildes, 
		mode_t mode
	); \end{CCode}

В mode можно указать следующие режимы: права на чтение, запись и исполнение для пользователя, группы и “остальных“, так и SUID, SGID и sticky bit. 

\begin{CCode}{Например}
	int chmod("/home/ag/myfile", S_ISUID | S_IWGRP); \end{CCode}

Установит SUID и даст группе право на запись.