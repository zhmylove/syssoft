Проверить права доступа к файлу можно с помощью системного вызова access (2). Его аргументы интуитивно понятны --- путь к файлу path и режим доступа amode.

\begin{CCode}{access (2)}
	int access(
		const char *path, 
		int amode
	); \end{CCode}

В amode можно указать следующие режимы:

	\begin{myenv}{R\_OK}{чтение;}
	\end{myenv}
	\begin{myenv}{W\_OK}{запись;}
	\end{myenv}
	\begin{myenv}{X\_OK}{исполнение;}
	\end{myenv}
	\begin{myenv}{F\_OK}{существование.}
	\end{myenv}

Cистемной вызов access (2) возвращает 0 в случае, если у файла установлены заданные в <amode> права, и код ошибки в ином случае. 

Также режимы можно склеивать через операцию дизъюнкции (|).

\begin{CCode}{Например}
	access(“/home/ag/myfile“, W_OK | X_OK); \end{CCode}

Вернет код успеха, в случае если файл “/home/ag/myfile“ доступен на запись и выполнение.

