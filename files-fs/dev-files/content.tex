Файлы устройств обычно предоставляют интерфейс для доступа к принтерам и прочим периферийным устройствам, но они также могут быть использованы для доступа к ресурсам на этих устройствах, например, разделам диска. Также они могут быть использованы для эмуляции доступа к системным ресурсам, которые не подключены к реальным устройствам, например, генератору псевдо-случайных чисел (/dev/urandom).

Для создания файлов блочных и символьных устройств можно использовать системный вызов mknod (2). Он позволяет сделать жесткую ссылку внутри файловой системы на файл, который будет сам по себе ссылаться на определенное устройство, используя механизм ссылки на устройство. Он принимает путь к файлу, желаемый режим досупа и непосредственно структура устройства.

\begin{CCode}{mknod (2)}
	int mknod( 
		const char *path, /* path to file */ 
		mode_t mode, 	  /* access mode */ 
		dev_t dev 		  /* device */
	); \end{CCode}

\begin{important}
	На самом деле этот системный вызов можно использовать и для создания файлов и именованных каналов.
\end{important}