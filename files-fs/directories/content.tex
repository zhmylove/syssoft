С точки зрения пользователя файлы в ОС UNIX организованы в виде древовидного пространства имен. Дерево состоит из ветвей (каталогов) начиная от (/) и заканчивается листьями (файлами).

Каталог отличается от обычного файла тем, что у него четко заранее заданная известная структура, которую можно представить в виде таблицы, содержащей имена находящихся в нем файлов и указатели на метаданные, позволяющие ОС производить действия с этими файлами.
