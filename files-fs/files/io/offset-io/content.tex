Мы уже написали о том, что есть простой ввод-вывод write и read, однако ему есть альтернатива --- ввод-вывод со смещением. 

Ввод-вывод со смещением позволяет не просто вводить и выводить в том месте, на которое указывает текущая позиция внутри файла, а позволяет осуществить ввод в конкретную позицию или прочитать оттуда данные.

Для его осуществления есть два системных вызова pread (2) и pwrite (2). Оба вызова принимают в качестве аргуметнов файловый дескриптор, буфер для чтения/записи, размер этого буфера и смещение относительно начала файла.

\begin{CCode}{pread (2)}
	ssize_t pread(
		int fildes, 	/* descriptor of file */
		void *buf,		/* buffer */
		size_t nbyte, 	/* size of buffer  */
		off_t offset	/* offset of start of file */
	); \end{CCode}
Возвращает количество считанных байт или код ошибки.

\begin{CCode}{pwrite (2)}
	ssize_t pwrite(
		int fildes, 		/* descriptor of file */
		const void *buf,	/* buffer */
		size_t nbyte,		/* size of buffer  */
		off_t offset		/* offset of start of file */
	); \end{CCode}
Возвращает количество записанных байт или код ошибки.

\textbf{Когда может понадобиться ввод или вывод со смещением?}

Если у вас есть однопоточное приложение, внутри открытого файла вы можете управлять указателем с помощью lseek. Если же у вас несколько потоков и они работают с одним и тем же файлом (например, большая база данных), а указатель все равно один, может произойти такая неприятная ситуация.

Допустим, вы поставили указатель на то место, куда хотели бы записать 0.5Гб, и начали запись. В это время другой поток что-то прочитал, то есть подвинул указатель. Теперь запись по этому указателю может привести к фатальным последствиям --- испортятся важные данные.

Системные вызовы выполняются атомарно. То есть, при выполнении любого системного вызова блокируется выполнение всего процесса целиком и полностью --- никто другой в этот момент не может сделать системный вызов из этого процесса. Это значит, что и pread и pwrite выполняются атомарно т.е. никто не сможет воспользоваться lseek и сдвинуть наш указатель. Именно этим и пользуются программисты, когда используют ввод и вывод со смещением.
