Подробнее о том как работает конец файла. Никакого магического символа конца файла нет. У файла есть inode, а внутри него есть размер файла, т.е. сколько байт внутри этого файла.

Если же мы попытаемся прочитать большее количество байт ОС, это понимает и возвращает то количество байт, которое мы можем прочитать. Причем мы помним, что внутри файла всегда есть указатель. Указатель всегда смещается при чтении или записи, и в какой-то момент нам вернутся оставшиеся к чтению байты, количество которых будет меньше размера буфера. Если мы попытаемся произвести чтение еще раз, ОС увидит, что указатель на конце файла, и вернет 0. Это будет являться признаком конца файла. 

\\
\textbf{Что есть символ конца файла? Что вводит нас в заблуждение?}

Мы знаем, что есть некий специальный символ, ввод которого внутрь терминала или внутрь эмулятора терминала, заставляет терминал сообщить ОС, что ввод закончен. То есть, если в терминале выполняется считывание с STDIN и мы что-то вводим, то последовательность ctrl + D даст ОС понять, что ввод (файл) закончен. Внутри файла этот символ не хранится нигде, это управляющий символ терминала для сообщения о конце ввода. Он никак не говорит о том, что файл закончился.
