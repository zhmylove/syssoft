Системный вызов lseek (2) позволяет оперировать с указателем внутри файла с помощью аргументов whence (действие с указателем) и offset (смещение позиции в байтах). Что позволяет производить дальнейшие операции чтения/записи с файлом, начиная с установленного lseek положения указателя. Этот системный вызов возвращает полученное смещение в байтах или код ошибки.

\begin{CCode}{lseek}
	off_t lseek( 
		int fildes,	 		/* opened file descriptor number */ 
		off_t offset, 		/* position offset (bytes) */ 
		int whence 	 		/* action to pointer */ 
); \end{CCode}

\begin{center}
\begin{tabular}{l|p{10cm}}
	\textbf{Значения whence}	&	\textbf{Назначение} \\
	\hline
	SEEK\_CUR	&	Указатель сместится на offset относительно своей текущей позиции внутри файла \\
	\hline
	SEEK\_SET	&	Указатель сместится на offset от начала файла \\
	\hline
	SEEK\_END	&	Указатель сместится на offset от конца файла \\
	\hline
	SEEK\_DATA	&	Перемещение указателя к началу следующей файловой дырки \\
	\hline	
	SEEK\_HOLE	&	Перемещение указателя к началу следующего сегмента данных
\end{tabular}
\end{center}

\textbf{В каких случаях может не получиться сместить указатель файла?}

Собственно, вариант того, что ОС откажется смещать указатель, только один --- если это файл, который не поддерживает указатель внутри себя. К таким файлам относится FIFO, символьные устройства, сокеты и символьные ссылки. 
