В классическом юниксе всего семь типов файлов, каждый из них имеет свои особенности и назначение. Кратко рассмотрим каждый из них, приведя их обозначения в скобках:

\begin{defi}{Обычные файлы или regular files (-)}	
	наиболее общий тип файлов. Содержит данные в некотором формате --- просто последовательность байт для операционной системы.
\end{defi}

\begin{defi}{Директории или directories (d)}
	файлы, содержащие имена находящихся в них файлов и указатели на метаданные, позволяющие ОС производить действия с этими файлами. Их удобно представлять в виде таблиц, в которых каждая строка содержит в себе жесткую ссылку.
\end{defi}

\begin{defi}{Символьные ссылки или symbolic links (l)}
	файлы, косвенно адресующие другие файлы файловой системы. Содержимое этих файлов интерпретируется как путь к файлу, на который они ссылаются.
\end{defi}

\begin{defi}{Символьные устройства (с) и блочные устройства (b)}
	два способа предоставить программам интерфейс для записи на какое-либо устройство. Канонически они различаются наличием или отсутствием буфера. 
	
	То есть в блочных устройствах (block devices) короткие порции данных склеиваются операционной системой в большие блоки, и обмен данными происходит уже этими самыми блоками. В символьных устройствах (character devices) обмен данными происходит посимвольно, то есть каждый символ будет отправляться сразу.
\end{defi}

\begin{defi}{Именованные каналы или named pipes (p)}
	файлы, используемые для передачи данных между процессами.
\end{defi}

\begin{defi}{Сокеты или sockets (s)}
	двусторонние каналы передачи данных между двумя единицами соединения.
\end{defi}

В \textbf{Solaris OS} также присутствует \textbf{тип файлов Doors} (двери). Они имеют \textbf{обозначение D} (не путайте с d у директорий) и нужны для осуществления межпроцессного взаимодействия.
