Теперь мы поговорим о том, что не особо соответветствует философии UNIX-way. UNIX-way --- это философия юникс, одно из положений которой гласит: делай вещь, которая работает просто и решай какую-то одну задачу. В противовес этому есть такие решения, как \textbf{fcntl f (file) cntl (control)} --- системный вызов, предназначенный для выполнения операций над файлом, используя его файловый дескриптор. Операция определяется аргументом cmd.

Собственно функция fcntl позволяет работать с файловыми дескрипторами. Первый аргумент и есть файловый дескриптор, с которым мы работаем. Дальше, выполнять некие команды с этим дескриптором, которые мы передаем вторым аргументом, а дальше переменное количество аргументов, которые будут параметризовать запрошенную команду. Возвращает соответствующее значение той команды, которую мы запросили.

\begin{CCode}{fcntl(2)}
	#include <sys/types.h>
	#include <unistd.h>
	#include <fcntl.h>

	int fcntl(
		int fildes, 
		int cmd, /* args */ ... 
	); \end{CCode}

\textbf{Управление флагами дескриптора файла}

\begin{myenv}{F\_GETFD}{получает значения флагов дескриптора, описанных в <fcntl.h>, для указанного файлового дескриптора.}
\end{myenv}

\begin{myenv}{F\_SETFD}{устанавливает значения флагов дескриптора для указанного файлового дескриптора, в соответствии с битами arg. Если флаг FD\_CLOEXEC равен 0, то файловый дескриптор останется открытым после вызова exec, иначе - будет закрыт.}
\end{myenv}
	
\textbf{Управление флагами состояния файла}

\begin{myenv}{F\_GETFL}{получает флаги статуса файла и режим доступа, для указанного файлового дескриптора.}
\end{myenv}

\begin{myenv}{F\_SETFL}{устанавливает флаги статуса файла, согласно значению arg. Оставшиеся биты (режим доступа, флаги создания) в значении arg игнорируются.}
\end{myenv}

\textbf{Управление сокетами}

\begin{myenv}{F\_GETOWN}{возвращает pid или pgid процесса, который получает сигнал SIGURG (появление срочных (urgent)) данных. Положительные значения относятся к pid, отрицательные, кроме -1 - к pgid.}
\end{myenv} 

\begin{myenv}{F\_SETOWN}{изменить pid или pgid процесса, ответственного за получение SIGURG. Положительное значение arg относится к pid, а отрицательные, кроме -1 - к pgid.}
\end{myenv}
