Для создания дубликата (копирования) существующего файлового дескриптора используются ситемные вызовы dup(2) и dup2(2) (от слова duplicate). Оба этих вызова в качестве первого аргумента принимают номер уже открытого файлового дескриптора. Разница между этими двумя вызовами лишь в том, что dup2 позволяет задать значение желаемого номера нового файлового дескриптора во втором агрументе.

\begin{CCode}{dup(2)}
	int dup( 
		int fildes /* opened file descritor number */
	); \end{CCode}
Возвращает: дубликат файлового дескриптора или код ошибки.

\begin{CCode}{dup2(2)}
	int dup( 
		int fildes /* opened file descritor number */, 
		int fildes2 
	); \end{CCode}
Возвращает: дубликат файлового дескриптора или код ошибки.

\textbf{В каких случаях может произойти ошибка?}

Очевидно, в случае если файловый дескриптор указывает несуществующий, не открытый поток. Если функция dup(2) не находит свободного места в таблице дескрипторов открытых файлов, то тоже вернется ошибка, что нет свободного места. 
