Как уже было сказано, сокеты --- это двусторонние каналы передачи данных между двумя единицами соединения. Существует несколько видов сокетов:

\begin{itemize}
	\item UNIX domain 
	\item Сетевые 
	\item BSD сокеты 
	\item Прочие
\end{itemize}

Они различаются по семействам адресов. 

Сокет подразумевает, что у вас есть некий сервер, который слушает определенный сокет.


\textbf{Для работы с сокетами существуют следующие системные вызовы.}

Cистемный вызов socket(2) создает новый сокет указанного типа и возвращает номер файлового дескриптора или код ошибки. В качестве аргументов этот системный вызов принимает семейство адресов, тип соединения и протокол.

\begin{CCode}{socket(2)}
	int socket( 
		int domain, 	/* address domain */ 
		int type,		/* type */ 
		int protocol	/* protocol (ex. tcp, udp) */ 
	); \end{CCode}

Системный вызов bind(2) используется на стороне сервера для того, чтобы ассоциировать сокет с адресом (описанным структурой address), например, адресом хоста и портом. Этот системный вызов возвращает 0 или код ошибки. Вызов принимает номер файлового дескриптора сокета, имя и длину поля имени.

\begin{CCode}{bind(2)}
	int bind( 
		int s, 							/* socket's descriptor number */ 
		const struct sockaddr *name, 	/* name */ 
		int namelen 					/* size of name in bytes */ 
	); \end{CCode}

Системный вызов listen(2) используется на стороне сервера для того, чтобы TCP сокет, связанный с адресом, начал “слушать” указанный порт. Системный вызов принимает номер файлового дескриптора сокета и длину очереди. Возвращает 0 или код ошибки.

\begin{CCode}{listen(2)}
	int listen( 
		int s, 			/* socket's descriptor number */ 
		int backlog 	/* length of queue */ 
	); \end{CCode}


Системный вызов accept(2) используется на стороне сервера. Принимает входящее соединение, создает новое TCP соединение и новый сокет, связанный с подключенным. Возвращает номер дескриптора или код ошибки.

\begin{CCode}{accept(2)}
	int accept( 
		int s, 					/* socket's descriptor number */ 
		struct sockaddr *addr,	/* client's address */ 
		socklen_t *addrlen 		/* size of client's address */ 
	); \end{CCode} 


Системный вызов connect(2) используется на стороне клиента и назначает свободный порт сокету. В случае TCP сокета пытается установить новое TCP соединение. Этот вызов принимает номер файлового дескриптора сокета, имя и длину поля имени. Он возвращает 0 или код ошибки.

\begin{CCode}{connect(2)}
	int connect( 
		int s,							/* socket's descriptor number */ 
		const struct sockaddr *name, 	/* name */ 
		int namelen 					/* size of name in bytes */ 
	); \end{CCode} 
