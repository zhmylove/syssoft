Операционная система предоставляет пользователю следующую функциональность:

\textbf{Многозадачность}

Позволяет одновременно выполнять несколько программ, изолируя их в разных адресных пространствах.

\textbf{Виртуализация памяти}

Операционная система абстрагирует программиста от физического адресного пространства, позволяет использовать больше адресного пространства, чем доступно оперативной памяти, а также изолирует программы, размещая их по непересекающимся адресам.

\textbf{Управление устройствами}

Система позволяет нам абстрагироваться от взаимодействия с устройствами, беря эту задачу на себя.

В самом деле, было бы сложно каждый раз самим думать о том как, например, спозиционировать читающую головку на жестком диске или как вывести что-то на принтер.

\textbf{Обработка прерываний}

Операционная система реагирует на события, которые являются внешними по отношению к процессу (например, запрос на ввод-вывод). Такие события называются прерываниями.

\textbf{Расширение набора операций, доступных программам}

Операционная система расширяет набор команд, доступных программам. На уровне операционной системы появляются системные вызовы --- обращения к функциям, которые реализованы в ядре операционной системы. 